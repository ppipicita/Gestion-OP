\documentclass[10pt]{beamer}

\usetheme[progressbar=frametitle]{metropolis}
\usepackage{appendixnumberbeamer}

\usepackage{booktabs}
\usepackage[scale=2]{ccicons}

\usepackage{pgfplots}
\usepgfplotslibrary{dateplot}

\usepackage{xspace}
\newcommand{\themename}{\textbf{\textsc{metropolis}}\xspace}

\title{Auxiliar 1}
\subtitle{Gestión de Operaciones Mineras}
\date{\today}
\author{Equipo Docente MI6081}
\institute{Universidad de Chile - FCFM}
% \titlegraphic{\hfill\includegraphics[height=1.5cm]{logo.pdf}}

\begin{document}

\maketitle

\begin{frame}[allowframebreaks]{Contenidos}
  \setbeamertemplate{section in toc}[sections numbered]
  \tableofcontents[hideallsubsections]
\end{frame}

\section{Introducción}

\begin{frame}[fragile]{Presentación}
Equipo Docente:
\begin{itemize}
    \item Profesor de Cátedra: Benjamín Galdames
    \item Auxiliares:
    \begin{itemize}
        \item Pía Iglesias
        \item Sebastian Reyes
        \item Héctor Alarcón
    \end{itemize}
\end{itemize}
Cátedra: Martes 18:00 a 21:00 hrs.\\
Auxiliar: Viernes 18:00 a 19:00 hrs.
\end{frame}

\begin{frame}[fragile]{Reglas del Curso}
\begin{itemize}
    \item Cátedras obligatorias (se tomará asistencia).
    \item Auxiliares no obligatorias (se tomará asistencia).
    \item Charlas de seguridad: 5 diapositivas aproximadamente, 10 minutos de exposición.
    \item Evaluaciones:
    \begin{itemize}
        \item Controles de lectura: Entre 2 a 4.
        \item Trabajos personales: 2.
        \item Examen obligatorio: Presentación grupal de caso de estudio.
    \end{itemize}
    \item Condiciones:
    \begin{itemize}
        \item Nota Final: $0.5Examen+0.5(0.2CL+0.8TP)$.
        \item $CL>=4.0$ y $TP>=4.0$.
        \item No hay CL ni TP recuperativos.
        \item No se borra la peor nota.
    \end{itemize}
\end{itemize}
\end{frame}

\end{document}
